
\usepackage[left=4.5cm, width=14.00cm, height=21.00cm]{geometry}
\usepackage[english]{babel}
\usepackage[utf8]{inputenc}
\usepackage[T1]{fontenc}
\usepackage{dsfont}
\usepackage{amsmath}
\usepackage{amsfonts}
\usepackage{amssymb}
\usepackage{graphicx}
\usepackage{paracol}
\usepackage{xparse}
\usepackage{sidecap}
\usepackage[makeroom]{cancel}
\usepackage{capt-of}
\usepackage{caption}
\usepackage[dvipsnames]{xcolor}
\usepackage{xpatch}
\usepackage{subcaption}
\usepackage[most]{tcolorbox}
\usepackage{lipsum}
\usepackage{float}
\usepackage{imakeidx}
\usepackage{wrapfig}
\usepackage{marginnote}
\usepackage{ upgreek }
\usepackage{bm}
\usepackage{enumerate}
\usepackage{mathrsfs} 

\graphicspath{{Images/}}


\makeindex[columns=3, title=Indice Analitico, intoc]

\captionsetup{font = {it, small}, labelfont={color=NavyBlue, bf}}



\newcommand{\kkt}{Karush-Kuhn-Tucker}
\renewcommand{\L}{\mathscr{L}}
\newcommand{\aL}{\mathscr{L}^{-1}}
\newcommand{\laplace}[1]{\mathscr{L} \left\{ #1 \right\}}
\newcommand{\antilaplace}[1]{\mathscr{L}^{-1} \left\{ #1 \right\} }
\newcommand{\re}[1]{\textrm{Re}\left(#1\right)}
\newcommand{\im}[1]{\textrm{Im}\left(#1\right)}
\newcommand{\lapint}{\int_{0^-}^\infty}
\renewcommand{\c}{^*}
\newcommand{\R}{\mathds R}
\newcommand{\vett}[1]{\boldsymbol{#1}}
\newcommand{\vstar}[1]{\vett{#1}^*}
\newcommand{\lag}{\mathcal L}
\newcommand{\de}[1]{\textbf{\textcolor{NavyBlue}{#1}}}
\newcommand{\pd}[2]{\frac{\partial #1}{\partial #2}}
\newcommand{\act}{\mathcal A}
%\renewcommand{\epsilon}{\varepsilon}
\newcommand{\eps}[1]{\epsilon_{#1}}
\newcommand{\figura}[5]{\begin{SCfigure}[#2][b!h!t!]
		\centering
		\includegraphics[width=#1 cm]{#3}
		\caption{#4} \label{#5}
\end{SCfigure}}
\newcommand{\fun}[1]{\mathcal{#1}}
\newcommand{\F}{\mathcal F}
\newcommand{\s}{^*}
\renewcommand{\matrix}[1]{\begin{bmatrix} #1 \end{bmatrix}}
\renewcommand{\vector}[1]{\begin{pmatrix} #1 \end{pmatrix}}
\newcommand{\zkm}{z_{k-\frac 12}}
\newcommand{\zkp}{z_{k+\frac 12}}
\newcommand{\argref}[1]{{\scriptscriptstyle  (#1) }}
\newcommand{\Lt}{\tilde{\mathcal L}}
\renewcommand{\H}{\mathcal{H}}
\newcommand{\B}{\mathcal{B}}
\newcommand{\bfcolor}[1]{\renewcommand*{\textbf}[1]{{\bfseries {\color{#1}##1}}}}

\setcolumnwidth{0.3\textwidth}



\newcounter{concetti}
\newenvironment{concetto}{
	
	\bfcolor{NavyBlue}
	\refstepcounter{concetti}
	{\color{NavyBlue}\textbf{Concetto \theconcetti:}} \quad
}{
}
\numberwithin{concetti}{chapter}
\tcolorboxenvironment{concetto}{
	boxrule=0pt,
	boxsep=0pt,
	colback={White!90!NavyBlue},
	enhanced jigsaw, 
	borderline west={2pt}{0pt}{NavyBlue},
	sharp corners,
	before skip=5pt,
	after skip=10pt,
	breakable,
}

\newcounter{teoremi}
\newenvironment{teorema}[2]{
	\bfcolor{ForestGreen}
	\refstepcounter{teoremi}
	\textbf{Concetto \theteoremi #1} 
	\vspace{3mm} 
	
	\texttt{Ipotesi: } #2
	
	\vspace{3mm} 
	
	\texttt{Enunciato}: 
}{
}
\numberwithin{teoremi}{chapter}
\tcolorboxenvironment{teorema}{
	boxrule=0pt,
	boxsep=0pt,
	colback={White!100!ForestGreen},
	enhanced jigsaw, 
	borderline west={2pt}{0pt}{ForestGreen},
	sharp corners,
	before skip=5pt,
	after skip=10pt,
	breakable,
}


\newenvironment{dimostrazione}{
	\bfcolor{Orchid}
	\textbf{Dimostrazione} \quad
}{ }
\tcolorboxenvironment{dimostrazione}{
	boxrule=0pt,
	boxsep=0pt,
	colback={White!100!NavyBlue},
	enhanced jigsaw, 
	borderline west={2pt}{0pt}{Orchid},
	sharp corners,
	before skip=5pt,
	after skip=10pt,
	breakable,
}

\newenvironment{osservazione}{
	\bfcolor{BurntOrange}
	\textbf{Osservazione: }
}{ }
\tcolorboxenvironment{osservazione}{
	boxrule=0pt,
	boxsep=0pt,
	colback={White!100!BurntOrange},
	enhanced jigsaw, 
	borderline west={2pt}{0pt}{BurntOrange},
	sharp corners,
	before skip=5pt,
	after skip=10pt,
	breakable,
}

\newenvironment{note}{
	\bfcolor{CadetBlue}
	\textbf{Note: }
}{ }
\tcolorboxenvironment{note}{
	boxrule=0pt,
	boxsep=0pt,
	colback={White!100!CadetBlue},
	enhanced jigsaw, 
	borderline west={2pt}{0pt}{CadetBlue},
	sharp corners,
	before skip=5pt,
	after skip=10pt,
	breakable,
}




\newcounter{numdem}
\numberwithin{numdem}{chapter}
\newenvironment{demonstration}{
	\noindent
	\refstepcounter{numdem}
	
%	\renewcommand{\de}[1]{ { \color{ForestGreen} \textbf{#1} } }
	
	{\color{ForestGreen}\textbf{Demonstration \thenumdem:}}
}{
}
\tcolorboxenvironment{demonstration}{
	boxrule=0pt,
	boxsep=0pt,
	colback={White!90!ForestGreen},
	enhanced jigsaw, 
	borderline west={2pt}{0pt}{ForestGreen},
	sharp corners,
	before skip=5pt,
	after skip=5pt,
	breakable,
}

\newcounter{exercises}
\numberwithin{exercises}{chapter}
\newenvironment{exercise}[1]{
	\bfcolor{Periwinkle}
	\noindent
	\refstepcounter{exercises}
	{\color{Periwinkle}\textbf{Exercise \theexercises#1}  } 
	\vspace{3mm}
	
	\noindent
}{
}


\tcolorboxenvironment{exercise}{
	boxrule=0pt,
	boxsep=0pt,
	colback={White!90!Periwinkle},
	enhanced jigsaw, 
	borderline west={2pt}{0pt}{Periwinkle},
	sharp corners,
	before skip=10pt,
	after skip=10pt,
	breakable,
}

\newcounter{examples}
\numberwithin{examples}{chapter}
\newenvironment{example}[1]{
	\bfcolor{Periwinkle}
	\noindent
	\refstepcounter{examples}
	{\color{Periwinkle}\textbf{Example \theexamples#1}  } 
	\vspace{3mm}
	
	\noindent
}{
}


\tcolorboxenvironment{example}{
	boxrule=0pt,
	boxsep=0pt,
	colback={White!90!Periwinkle},
	enhanced jigsaw, 
	borderline west={2pt}{0pt}{Periwinkle},
	sharp corners,
	before skip=10pt,
	after skip=10pt,
	breakable,
}
