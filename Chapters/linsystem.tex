\chapter{Linear Systems with N Degrees of Freedom}
	Studying vibrations relates to the study of the system by building it's \de{mathematical model} based on \textbf{physical property} and the \textbf{accuracy} wanted in the description of the system itself (it's not necessary to describe the system \textit{at atomic detail} if we are only interested in the gross motion).
	
	\textbf{AGGIUNGERE ESMPIO DEL PENDOLO}
	
	Models can be arbitrarily \textit{simple} or \textit{complex} that depends on a various number of \textbf{parameters} and \textbf{degrees of freedom}. In particular \de{dynamical systems} ca be classified
	\begin{itemize}
		\item as \textbf{linear} or \textbf{non-linear}: such property is a matter of model decision and no system is linear in reality. In general for \textit{low displacements} from the equilibrium the linear model converges to the true non-linear behaviour;
		
		\item depending on the inputs that can be \textbf{deterministic} or \textbf{random} (this will require a statistical description of the system).
	\end{itemize}
	
	
	
	
	
	
	
	
	
	
	
	
	
	
	
	
	
	
	
	
	
	
	
	
	
	
	
	
	
	
	