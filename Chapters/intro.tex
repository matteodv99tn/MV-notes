\chapter{Introduction}
	The aim of this course is understanding how mechanical systems moves while subjected to external forces. Generally \textbf{dynamical systems} (mechanical, but also electrical, biological...) can so be regarded as \textit{black box} where, for some given \textbf{inputs} $u(t)$ (independent variable) function of time, it produces an \textbf{output} $y(t)$ (dependent variable) that's still depend on time.
	
	\textbf{Vibrations} are a subset of the all allowable motion of the system that are characterized by a confined motion respect to the equilibrium point with a certain periodicity.\\
	Not all system are allowed to vibrate; from a formal point of view system vibrates if they are allowed to continuously reconvert potential energy (mainly elastic or gravitational) into kinetic. Vibration can be avoided by having \textit{components} (in general) that allow to dissipate energy reducing so the vibrational motion respect to the equilibrium position.
	
	\vspace{3mm}
	
	The study of vibration relates to the study of the system by building it's \textbf{mathematical model} that's based both on \textbf{physical properties} as well as the related \textbf{accuracy}. The same system can be in fact modelled with different \textit{depth} in terms of knowledge and detail: any addition in terms of physical phenomena (by increasing degrees of freedoms of the system, the number of parameters and equations) increases the complexity in the calculations (in simple words \textit{we have to handle more stuff}) but also allows to create a model that better describes the reality. From an engineering point of view this is a trade-off between complexity and accuracy and such balance strictly depends on the goal of the study.
	
	\vspace{3mm}
	
	Dynamical system can be classified according to multiple orthogonal definitions such:
	\begin{itemize}
		\item the type of input can be deterministic (described by a analytical function) or random (where so a statistical description of the system is necessarily needed);
		\item depending on the \textbf{linearity} (or not) of the system. In practise no system is purely linear however such behaviour can be approximated as valid for \textit{low displacement} from the equilibrium position where the two motions converges.
	\end{itemize}
	
	
	
	
	
	
	
	
	
	
	
	
	